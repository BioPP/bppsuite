\documentclass{article}
\usepackage{indentfirst}
\usepackage[utf8]{inputenc}
\usepackage[english]{babel}
\usepackage{graphicx}
\usepackage{url}

\title{Tutorial for bppsuite}

\addtolength{\topmargin}{-2cm}
\setlength{\textheight}{24cm}

\date{\today}

\author{Laurent Guéguen \\ {\small (laurent.gueguen@univ-lyon1.fr)}}

\begin{document}

\maketitle
\thispagestyle{empty}

\medskip 


In this tutorial, we will see how simulation of sequence evolution can
be performed using bio++ libraries and \texttt{bppseqgen} program, on
which all information is available there:

\url{https://github.com/BioPP/bpp-documentation/wiki}

\medskip

All the options are described in bppsuite manual:

\url{https://pbil.univ-lyon1.fr/bpp-doc/bppsuite/bppsuite.html}

We will specifically explore the simulation of protein algnments, but
those approaches are also available for nucleotide and codon
alignments.

A simulation follows a subtitution process, which is defined by
a (set of) substitution models, a tree, and in option a root
distribution and a substitution rate distribution.

\begin{itemize}
\item
  Many amino acid models are available in bio++, see
  \url{https://pbil.univ-lyon1.fr/bpp-doc/bppsuite/bppsuite.html#Protein},
  and we will more and more complex ways to perform simulartions
  followign sich models.

\item In all simulations, a tree is needed to guide the evolution
  process. First, we consider that the tree is in file
  \texttt{../../Data/LSUrooted.dnd}.

  
\end{itemize}

In the next sections, we will show more and more complex ways of
simulation, using more and more information from given data.


\section{Homogeneous in time and space}

First, we describe examples where the modeling defined the same on all
branches and sites.

\subsection{Homogeneous simple model}

In \verb|simple_hom.bpp|, we consider the simplest case of evolution:
homogeneous, stationnary with a simple model
WAG01\footnote{\label{WAG01}\url{https://pbil.univ-lyon1.fr/bpp-doc/bpp-phyl/html/classbpp_1_1WAG01.html#details}},
following a tree in file \texttt{LSU.dnd}. Both are gathered in
process \texttt{process1}. \texttt{simul1} sets the simulation, using
this process, of a 300 amino acids protein alignment.

\begin{verbatim}
  bppseqgen param=simple_hom.bpp
\end{verbatim}

In \texttt{simul1}, the argument \texttt{output.internal.sequences =
  true} means that simulated ancestral sequences are output, otherwise
there are only leaves sequences. The sequences are labeled with the
number of the node in the tree.

\subsection{Homogeneous mixture model}

Instead of using a same model for all sites, it has been shown more
accurate to mixtures of models, defining either amino acid properties
or profiles. In the next examples, we use the models
LGL08\_CAT\footnote{\url{https://pbil.univ-lyon1.fr/bpp-doc/bpp-phyl/html/classbpp_1_1LGL08__CAT.html#details}.}
that use a given number of amino acids profiles as found in databases.

In addition, we introduce a macro (NBCAT) that can be set from the
command line:

\begin{verbatim}
  bppseqgen param=mixed_hom.bpp NBCAT=40
\end{verbatim}

In this file, two simulations are performed. 

There are many ways to organize the mixing of models in a tree,
depending how the choice of a model on a branch depends on the choice
on the upper branch. In file \texttt{mixed\_hom.bpp}, the line

\begin{verbatim}
  scenario1= split(model=1)
\end{verbatim}

defines a simple \textbf{scenario} of organization of the models: a
site will follow the same model all along the tree, which with CAT
means the same profile. \texttt{process1} follows this scenario, for
\texttt{simul1}.


Without this option, at each node the process can switch freely
between submodels of the mixture. \texttt{process2} defines such a
process, and is used in simulation \texttt{simul2}.

Much more elaborate scenarios can be defined with the definition of
\textbf{paths} (see below?)


\subsection{Substitution rates}

In the previous examples, all sites evolve at the same rate, whereas
we could wish an heterogeneity in these rates. This can be done in two
ways, either on a globally or site specifically (see
section~\ref{sec:non-homog-space}).

In this example, we define a distribution of substitution rates:

\begin{verbatim}
  rate_distribution1 = Invariant(p=0.15,dist=Gamma(n=4))
\end{verbatim}

which is here a discrete Gamma distribution with 4 classes and a
probability $0.15$ of constant sites (null rate).


\subsection{Root frequencies}

In the previous descriptions, the modeling is stationary, which means
that the states distribution at the root is the stationary
distribution of the model.

It is possible to set specific root frequencies (or states, see below)
to define non-stationary process. In file \verb|mixed_IG_hom.bpp|, a
\verb|root_freq| is defined, and used in the definition of
\texttt{process2} as the root frequencies, which will be used for
\texttt{simul2}.


Both simulations can be run with command:

\begin{verbatim}
  bppseqgen param=mixed_IG_hom.bpp NBCAT=40
\end{verbatim}


\section{\label{sec:non-homog-space}Non-homogeneous in space}

In the previous modelings, the process is defined similarly for all
sites (even though with mixture models a site-specific choice of
submodels is done). We give how to define site-specificity, on the
root states, on the substitution rates, then on the models.

This is independent of modeling time non-homogeneity, so both can be
used together. 


\subsection{Site-specific root states}

States at the root of the tree can be set or sampled. In the tsv file
\verb|infos_rates.tsv|, column \texttt{States} assigns states to sites
at root. In \verb|sites_rate_state_hom.bpp|, this information is used
for \texttt{simul1}. The column of states can be named differently, as
long as it fits with \texttt{input.infos.states}.


Another way to define states at the root is to use an alignment from a
given file. Still in \verb|sites_rate_state_hom.bpp|,
\texttt{input.data1} is an alignment read in file \texttt{LSU.phy},
from which a random sequence is used as a root in \texttt{simul2}.


If the simul uses a process with defined \verb|root_freq|, the states
defined with \texttt{input.infos.states} are prioritary.


\subsection{Site-specific substitution rates}

Rates can also set up in a site-specific way, in a similar tsv file.
For example in column \texttt{Rates} of file \verb|infos_rates.tsv|,
middle sites are slower than extreme sites, as used in \texttt{simul3}
of \verb|sites_rate_state_hom.bpp|.

If the simul uses a process with defined \verb|rate|, the rates
defined with \texttt{input.infos.rates} are prioritary.


The rates can be defined without the root states (in which case those
states are sampled from a distribution). In \texttt{simul4}, they are
taken from a sequence in \texttt{input.data1}. In this case, only the
first 981 sites are considered (to fit with the number of rates in
\verb|infos_rates.tsv|). 


\begin{verbatim}
  bppseqgen param=sites_state_rate_hom.bpp NBCAT=40
\end{verbatim}

still with the site-specific rates defined in \verb|infos_rates.tsv|.

\subsection{Site-specific processes}

Beyond site-specific rates and root states, it is also possible to set
up site-specific processes. In file \verb|partition.bpp|, two simple
processes are defined, following the same stationary WAG01 model but
different trees. Then they are assigned to different sets of sites, in
a new process, which is used for simulation.

\begin{verbatim}
  bppseqgen param=partition.bpp
\end{verbatim}

Instead of preset boundaries in the partition, processes can be put
into a markovian chain along the sequence, or more simply with
autocorrelation probabilities of successive presence, as in
\verb|autocorr.bpp| for the same processes as before.


\section{Non-homogeneous in time}

In addition to the non-homogeneity in space, it is possible, totally
independently, to model non-homogeneity in time, which means that
different models can be assigned to different branches of the tree.


\subsection{Assigned non-homogeneity}
  
Still in file \verb|nonstat_WAG_nonhom.bpp|, \texttt{process1} in a
non-homogeneous process made of two WAG01+F\footref{WAG01} with
different equilibrium frequencies, assgined to two sets of branches,
and specific root frequencies.

\texttt{simul1} performs simple simulations with this model, whereas
\texttt{simul2} in addition uses site-specific rates

\subsection{Non-homogeneous mixtures}

This can be done also with mixture models, and all possibilities of
usage of submodels can be set within scenarios. In
\verb|nonstat_CAT_nonhom.bpp|, two CAT models are used, and with the
description:

\begin{verbatim}
  scenario1= split(model=(1,2))
\end{verbatim}

all paths where a same submodel of each CAT model is used on all
branches where this model is set. For example, with command:

\begin{verbatim}
  bppseqgen param=nonstat_CAT_nonhom.bpp NBCAT=10
\end{verbatim}

all 100 possible paths are with 2 submodels of CAT 10 are possible.

\section{Posterior simulations}

In all previous examples, simulations were done given models, root
frequencies, branch lengths. But in the prospect of simulations as
real as possible, it is difficult to choose without information about
real data. In bio++, all those parameters can be estimated through
maximum likelihood inference (via \texttt{bppml}), and given as inputs
for simulations. But because of the numerous hypotheses set in the
modeling (such as site-independence, too simple models, etc), the
resulting simulations can still be very unrealistic.

Another way to include data information in the simulation is to use
posterior transition probabilities. On a branch and a site, if the
used model is $M$ and the data $D$, for two states $x$ and $y$,
instead of using $P(y|x,M)$ as transition probabilities, we use
$P(y|x,M,D)$.

Hence site and branch specific information is taken into account, even
if it is not explicitly modeled.

In bio++, the \texttt{phylo} object links a \texttt{process} with
\texttt{data}, providing a likelihood. To simulate alignments
a posteriori, just switch in \texttt{simul} description the reference
to \texttt{process} with a reference the \texttt{phylo}.

In a complete posterior simulation, the sequences simulated at the
leaves will be exactly those of the given alignment, which is not very
interesting. So it is possible to release this to use directly the
models (and the prior probabilities) on some branches. In
\texttt{simul}, argument \texttt{nullnodes} accepts a list of branch
numbers, where the data will not be taken into account.

In the \verb|simple_data.bpp| file, \texttt{nullnodes} is set to
shortcut \texttt{Leaves}, which means that the simulation on all
external branches is following the model.

\begin{verbatim}
  bppseqgen param=simple_data.bpp
\end{verbatim}


Beware that such simulations takes more time than previouly, since all
the conditional likelihoods have to be computed.


\end{document}

